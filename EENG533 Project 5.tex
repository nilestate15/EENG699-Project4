\documentclass[12pt]{article}

\usepackage[
   paperheight = 11in,
   paperwidth = 8.5in,
   margin = 1in,
   footskip = 0.45in] {geometry} %  custom paper size and margins
\usepackage{graphicx}
\usepackage{float}
\usepackage{adjustbox}
\usepackage{color} %  custom colors
\usepackage{multicol}
\usepackage{enumitem}
\usepackage{booktabs}
\usepackage{hhline}
\usepackage{titlesec}
\usepackage{hyphenat}
\usepackage{array}
\usepackage{color} %  custom colors
\usepackage{graphicx}
\usepackage{listings, upquote, textcomp} %  "\lstinline{ }" for inline code
\usepackage[
   colorlinks = true,
   linkcolor = darkblue,
   urlcolor = darkblue,
   hypertexnames = false]{hyperref}

\setlength{\tabcolsep}{10pt}
\def\arraystretch{1.2}
\definecolor{gray}         {rgb}{0.50,0.50,0.50}
\definecolor{darkblue}     {rgb}{0.00,0.00,0.50}
\renewcommand{\labelenumi}{\arabic{enumi}.}
\renewcommand{\labelenumii}{\alph{enumii}.}
\setlength{\parindent}{0em}
\setlength{\parskip}{10pt plus 2pt minus 2pt}
\titlespacing\section{0pt}{10pt plus 2pt minus 2pt}{0pt plus 2pt minus 2pt}
\raggedbottom
\newcolumntype{L}[1]{>{
   \raggedright\let\newline\\\arraybackslash\hspace{0pt}}p{#1}}
\newcolumntype{C}[1]{>{
   \centering\let\newline\\\arraybackslash\hspace{0pt}}p{#1}}
\newcolumntype{R}[1]{>{
   \raggedleft\let\newline\\\arraybackslash\hspace{0pt}}p{#1}}
\lstdefinestyle{lstMat} {
   language = matlab,
   backgroundcolor = {},
   breakatwhitespace = true,
   breaklines = true,
   postbreak = \space,
   breakindent = 4ex,
   showstringspaces = false,
   basicstyle = %
      \lst@ifdisplaystyle\small\fi
      \ttfamily, %  48 chars in 3.5 in
   commentstyle = \color{gray},
   stringstyle = \color{red},
   keywordstyle = \color{blue},
   tabsize = 3,
   upquote = true}
\lstset{style = lstMat} %  Default syntax.

\definecolor{black}        {rgb}{0.00,0.00,0.00}
\definecolor{blue}         {rgb}{0.00,0.00,1.00}
\definecolor{azure}        {rgb}{0.00,0.40,1.00}
\definecolor{cyan}         {rgb}{0.00,0.80,1.00}
\definecolor{aqua}         {rgb}{0.00,0.80,0.50}
\definecolor{green}        {rgb}{0.00,0.60,0.00}
\definecolor{lime}         {rgb}{0.50,0.80,0.00}
\definecolor{yellow}       {rgb}{1.00,0.80,0.00}
\definecolor{orange}       {rgb}{1.00,0.40,0.00}
\definecolor{red}          {rgb}{1.00,0.00,0.00}
\definecolor{pink}         {rgb}{1.00,0.00,0.50}
\definecolor{magenta}      {rgb}{1.00,0.00,1.00}
\definecolor{purple}       {rgb}{0.50,0.00,1.00}
\definecolor{darkblue}     {rgb}{0.00,0.00,0.50}
\definecolor{gray}         {rgb}{0.50,0.50,0.50}
\definecolor{lightgray}    {rgb}{0.80,0.80,0.80}
\definecolor{shade}        {rgb}{0.95,0.95,0.95}
\definecolor{white}        {rgb}{1.00,1.00,1.00}
\newcommand{\textbk}[1]    {{\color{black}{#1}}}
\newcommand{\textbl}[1]    {{\color{blue}{#1}}}
\newcommand{\textaz}[1]    {{\color{azure}{#1}}}
\newcommand{\textcy}[1]    {{\color{cyan}{#1}}}
\newcommand{\textaq}[1]    {{\color{aqua}{#1}}}
\newcommand{\textgr}[1]    {{\color{green}{#1}}}
\newcommand{\textlm}[1]    {{\color{lime}{#1}}}
\newcommand{\textyw}[1]    {{\color{yellow}{#1}}}
\newcommand{\textor}[1]    {{\color{orange}{#1}}}
\newcommand{\textrd}[1]    {{\color{red}{#1}}}
\newcommand{\textpk}[1]    {{\color{pink}{#1}}}
\newcommand{\textmg}[1]    {{\color{magenta}{#1}}}
\newcommand{\textpu}[1]    {{\color{purple}{#1}}}
\newcommand{\textdb}[1]    {{\color{darkblue}{#1}}}
\newcommand{\textgy}[1]    {{\color{gray}{#1}}}
\newcommand{\textlg}[1]    {{\color{lightgray}{#1}}}
\newcommand{\textsh}[1]    {{\color{shade}{#1}}}
\newcommand{\textwh}[1]    {{\color{white}{#1}}}

\begin{document}
%  Header
\begin{minipage}{0.9\textwidth}
   \raggedright
   \large \textbf{\textsf{{\color{gray}EENG 533: Navigation Using GPS} \\
      Project 5: Introduction to GPS Processing}}
\end{minipage}
\vspace{1cm}

\section*{\textsf{Objectives}}

\begin{enumerate}
   \item Gain familiarity with GPS data formats
   \item Gain familiarity with how to download GPS data from the web
   \item Learn how to examine and plot relevant GPS data
\end{enumerate}

\section*{\textsf{Overview}}

In this lab, you will download real GPS data and start taking a look at what is
available with a standard, commercial-grade dual frequency receiver.

\section*{\textsf{Collaboration}}

This is an individual lab.   You are allowed to discuss any aspect of the lab
with other students, and you may look at each other's source code for debugging
purposes.  However, your programming must be your own; you may not copy or
transcribe someone else's program, in part or in whole.

\section*{\textsf{Provided Functions}}

This project will again make use of the georinex (\url{https://pypi.org/project/georinex/})
 Python library.  You should have it installed into your Python environment, 
but reference the project 2 description for installation instructions if needed.  

The \lstinline{datetime2tow()} function will again be used from the provided 
\lstinline{helper.py} file to convert from the calendar date and time given in 
a RINEX observation file to time of week in seconds.  The code template file 
(\lstinline{project5_template.py}) shows an example of how to convert all the 
RINEX observation times in a compact way.

%You are given the \lstinline{convert_rinex} MATLAB routine for use in this (and
%other) labs.  It converts RINEX-formated GPS data to a space-delimited
%spreadsheet of numbers.  This function can either be called by specifying the
%input and output file names:
%\begin{lstlisting}
%   convert_rinex('flwe1390.20o','flwe1390.mea');
%\end{lstlisting}
%or by giving no inputs, in which case all files ending with `.20o' will be
%converted:
%\begin{lstlisting}
%   convert_rinex();
%\end{lstlisting}

Once you have loaded the observation file and perfomed the time conversion, 
you will have access to the following values

%The output is a text file of space-delimited numbers whose columns are
\begin{center}
   \begin{tabular}{c|c|c} %  <- justification (lrc)
      Name & Description & Units \\
      \hline
         time\_tow & time since start of week  & sec \\
         sv.data & GPS pseudo-random number      & `Gxx' \\
         C1  & L1 C/A pseudorange        & m \\
         C2  & L2C pseudorange        & m \\
         L1  & L1 carrier phase          & cycles \\
         L2  & L2 carrier phase          & cycles \\
         P1  & L1 P-code pseudorange     & m \\
         P2  & L2 P-code pseudorange     & m \\
         S1  & L1 signal strength        & dB-Hz \\
        S2  & L2 signal strength        & dB-Hz
   \end{tabular}
\end{center} %  <- "*}" for full width

\section*{\textsf{Obtaining Data}}

For this project, you will be downloading data from the Continuously Operating
Reference Station (CORS) network.  The data file access page for this system can be found at
\url{https://geodesy.noaa.gov/UFCORS/}

Select a `Start Date' at least one day in the past so a full day of data is available.  
You can leave `Start Time' at 00:00.  For `Duration in
Hour(s)', pick 24.  Select a `Site ID'.  You can use the `CORS Map' to help or
just pick a site from the list.  Once you pick a site, the `Available Satellite
Systems' option will populate.  Pick `Legacy Applications / GPS (L1+L2 only)'.
Select a `Sampling Rate' and pick 15 seconds.  You do not need to request any Optional Files.  Just
click `get CORS data file'.

You should get a download of zip file.  Unzip the file and look for a file
ending the extension `.21o'.  This is the RINEX observation file.

\section*{\textsf{Task A}}

For this task, you should download 24 hours of data from somewhere in Florida
(your choice of the site), on a day of your choosing.  

Assuming you have loaded the observations into \lstinline{obs}, you can access the measurements
with \lstinline{obs['L1']} for example to get all L1 carrier phase measurements for all satellites and times.
\lstinline{obs['L1'].sel(sv='G16')} would give those measurements for just GPS satellite 16.  For some 
times and satellites a measurement may have a value of \lstinline{nan} which stands for ``not a number".
This means a specific measurement type was not available for this satellite at a specific time, most
likely due to the satellite being below the visible horizon.

%Convert the RINEX file
%using \lstinline{convert_rinex.m}.  You can then load the data into matlab
%using the load command:
%\begin{lstlisting}
%   data = load('my_file.mea');
%\end{lstlisting}

%You should then generate the following plots.  For all plots, do not plot values
%that are zero (which indicate measurements that do not exist).  You can avoid
%plotting zero simply by scaling the y axis to cover the actual data of interest.
%For example, to scale the y axis to the range 10 to 50, use
%\lstinline{ylim([10, 50]);}.

You should then generate the following properly labled plots.  The non-existent \lstinline{nan} measurements
will be correctly not plotted when using \lstinline{time_tow} for your x-axis.  The complete list
of satellites in the observation file are accessible as strings in \lstinline{obs.sv.data}.


\begin{enumerate}
   \item A plot of the visible GPS satellite PRNs vs. time in GPS \textbf{week seconds}
      (Hint 1: plot the PRNs vs. time using points.  This will prevent extra connecting lines
     from being added across time segments of \lstinline{nan}s.  Hint 2: a measurement divided by 
     itself times the PRN plotted vs \lstinline{time_tow} will produce a horizontal line with the value of
    the PRN, except for when a \lstinline{nan} is present)
   \item A plot of the visible GPS satellite PRNs vs. time in GPS \textbf{hour of day}
   \item A plot of all L1 C/A code pseudoranges (C1) vs. time in GPS hour of day
   \item A plot of all L1 Carrier-Phase measurements (L1) vs. time in GPS hour
      of day
   \item A plot of the time derivative of L1 C/A code pseudorange (C1) vs. time
      in GPS hour of day for just one satellite.  The time derivative should be
      expressed in units of meters/second (NOT meters/hour).  Choose a satellite
      that is available for a long time.  (Hint: use the numpy \lstinline{diff()} method
      to calculate the difference between pseudorange measurements and divide by the time interval)
%   \item A plot of the L1 Doppler measurements (D1) for the same satellite as in
%      plot 5  (Note, that your file might not have doppler data and
%      \lstinline{convert_rinex.m} will tell you so.)
   \item A plot of L1 C/A code pseudorange (C1) vs. time in GPS hour of day for
      the same satellite as plot 5
   \item A plot of L1 Carrier-Phase measurement (L1) vs. time in GPS hour of day
      for the same satellite as plot 5
   \item A plot of L2 Carrier-Phase Measurements (L2) vs. time in GPS hour of
      day for the same satellite as plot 5
   \item A plot of L2 P code pseudorange (P2) vs. time in GPS hour of day for
      the same satellite as plot 5
   \item A plot of all L1 C/A code measurements (C1) (converted to units of cycles,
      using the L1 wavelength) minus all L1 carrier-phase measurements (L1) vs.
      time.  Note that wavelength can be found by dividing the speed of light by
      the carrier frequency.
\end{enumerate}

Answer the following questions:
\begin{enumerate}[label=\Alph*.]
   \item What is the relationship between L1 C/A code pseudorange (C1) (figure
      6) and L2 P code pseudorange (P2) (see figure 9)?  (Note: you may need to
      zoom in or plot the difference in order to compare the two.)
   \item What is the relationship between the carrier-phase measurements and the
      pseudorange measurements?  Identify where there is a cycle slip (discontinuity) in plot 10.
      There should be some relatively ``flat'' areas in plot 10.  Zoom in on
      one.  How much does the code-minus-carrier that you plotted vary when
      there is no cycle slip?
   \item Based on plot 5, what is the range of velocity of the satellite
      relative to the stationary receiver that you observe?
   \item How many unique satellites were visible throughout the day?
\end{enumerate}

\section*{\textsf{Task B: Compare Satellite Visibilities}}

Now, download data from three additional sites: another site in Florida, a site
in Ohio, and a site that is very far away from Florida (Alaska or farther).  You
now have two Florida data files, an Ohio data file, and one far-away data file.

Generate a single plot that shows visible PRN vs. time in GPS hours for all four
locations.  Use a different color for each site's data set.  You should plot as
points, and you should use small vertical offsets in order to be able to see all
four receivers.  The point is to easily compare when each satellite was visible
for all three receivers on a single plot.  In one or two sentences, briefly
comment on what you see in this plot.

An example of what this figure should look like is shown here:
\begin{figure}[H] %  <- "*}" for full width; "[H]" to not float
   \centering
   \includegraphics %  <- add "[angle=90,origin=c]" to rotate
   {example.pdf} %  <- name of file here
   \caption{An example of figure 2: (blue) Florida 1, (cyan) Alaska, (green)
      Ohio, and (yellow) Florida 2.} %  <- remove for no caption
   \label{fig_example}
\end{figure} %  <- "*}" for full width

\section*{\textsf{Deliverables}}

Submit via Canvas your Python script and a pdf file with your
plots and answers to the questions from Task A and B.

\section*{\textsf{Grading}}

You will be graded for completeness and reasonableness of your answers.  Points
will be deducted for plots that are not properly labeled.

\end{document}
